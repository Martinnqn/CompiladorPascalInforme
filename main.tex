\documentclass[a4paper]{report}
\usepackage[utf8]{inputenc}
\usepackage[spanish, es-tabla]{babel}
\usepackage[]{enumerate}
\usepackage[]{amssymb}
\usepackage[colorlinks=true,linkcolor=black,citecolor = blue, urlcolor = blue]{hyperref}
\usepackage{graphicx}
\usepackage{float}
\usepackage{multirow}
\usepackage{multicol}
\usepackage{listings}
\usepackage{subfigure}
\usepackage{pgfplots}
\usepackage{mathbbol}
\usepackage{amsmath}
\usepackage{color}
%para gramatica bnf
\usepackage{bnf}
\pgfplotsset{compat=1.14}
%para margen
\usepackage[a4paper,left=2cm,right=2cm,top=3cm,bottom=3cm]{geometry}
\usepackage{geometry}
%\usepackage{scrextend}
%para el automata
\usepackage{tikz}
\usetikzlibrary{shapes}
\usetikzlibrary{shapes.geometric}
\usetikzlibrary{automata, positioning} %para notacion de automata
\usetikzlibrary{babel} %compatibilidad con babel
%para el arbol de directorios
\usepackage{dirtree}
%para codigo con color
\usepackage{minted}
%para centrar minted (no permite mostrar número de líneas de código)
%\RecustomVerbatimEnvironment{Verbatim}{BVerbatim}{} 
%para el arbol if_ambiguo hecho en dot.
\usepackage[pdf]{graphviz}
\usepackage{psfrag}
%para declarar nuevos elementos float 
\usepackage{newfloat}

\usepackage[export]{adjustbox}[2011/08/13]

\usepackage{changepage}
\usepackage{bm}
\usepackage{xcolor,colortbl}

\newtheorem{definition}{{\bf\sc Definición }}[section]
\newtheorem{theorem}{Teorema}[section]

%para las listas "mylist" personalizadas con caption list
\DeclareFloatingEnvironment[placement={!ht},name=Lista]{mylist}

%%%%%%%%%%%%%%%%%%%%%%%%%%%%%%%%%%%%%%%%%%%%%%%%%%%%%%%%%%%%%%%%%
%% The following definitions are to extend the LaTeX algorithmic 
%% package with SWITCH statements and one-line structures.
%% The extension is by 
%%   Prof. Farn Wang 
%%   Dept. of Electrical Engineering, 
%%   National Taiwan University. 
%% 
\newcommand{\SWITCH}[1]{\STATE \textbf{switch} (#1)}
\newcommand{\ENDSWITCH}{\STATE \textbf{end switch}}
\newcommand{\CASE}[1]{\STATE \textbf{case} #1\textbf{:} \begin{ALC@g}}
\newcommand{\ENDCASE}{\end{ALC@g}}
\newcommand{\CASELINE}[1]{\STATE \textbf{case} #1\textbf{:} }
\newcommand{\DEFAULT}{\STATE \textbf{default:} \begin{ALC@g}}
\newcommand{\ENDDEFAULT}{\end{ALC@g}}
\newcommand{\DEFAULTLINE}[1]{\STATE \textbf{default:} }
%% 
%% End of the LaTeX algorithmic package extension.
%%%%%%%%%%%%%%%%%%%%%%%%%%%%%%%%%%%%%%%%%%%%%%%%%%%%%%%%%%%%%%%%%


\begin{document}

\title{Desarrollo de un Compilador para Pascal}
\author{Bermudez Martín \and Marinelli Giuliano}
\date{Universidad Nacional del Comahue}

\titlepage

\begin{center}
	
	\vspace{-0.5cm}
	
	{\Large{\bf \sc Universidad Nacional del Comahue}}\\
	
	{\Large { \sc Facultad de Informática}}\\
	
	\vspace{-2.5cm}
	\mbox{\includegraphics[width=2.5cm,height=2.5cm]{img/unc.png}\hspace{12cm} \includegraphics[width=2.5cm,height=2.5cm]{img/fai.png}}
	
	
	\vspace{2cm}
	
	\ \\
	\ \\
	%titulo
	{\Large {\bf Diseño de Compiladores e Intérpretes}} \\ 
	\vspace{1cm}
	%subtitulo
	{\LARGE {\bf Desarrollo de un Compilador para Pascal}}\\
	\vspace{3cm}
	
    {\Large GRUPO 1}\\
    \vspace{1cm}
	
	{\Large Bermudez Martín}\\
    \vspace{0.5cm}
    {\Large FAI-1140}
	\vspace{1cm}
	
	{\Large Marinelli Giuliano}\\
    \vspace{0.5cm}
    {\Large FAI-1196}
	\vspace{5cm}
	
	{\Large \{martin.bermudez, giuliano.marinelli\}\\
    @est.fi.uncoma.edu.ar}\\
	\vspace{1cm}
    
	\vfill
	{\Large {\sc Neuqu\'en}\hspace{6cm}{\sc Argentina}}\\
	\ \\
	
	{\Large 2018}\\
	
\end{center}

\pagebreak

\tableofcontents
\thispagestyle{empty}

%\begin{adjustwidth*}{3cm}{3cm}

\chapter{Introducción}

\section{Compiladores y proceso de desarrollo}
Los compiladores son programas que permiten la traducción entre especificaciones de lenguajes de programación. Tienen el objetivo principal de traducir lenguajes de alto nivel en otros de más bajo nivel que sean interpretables por la arquitectura del hardware en que se ejecutarán.

Para poder lograr esta tarea, los compiladores realizan varios procesos, donde, en un principio, se tiene como entrada el código del lenguaje a traducir el cual pasará por las etapas que pueden verse en la figura \ref{fig:etapas} y finalmente retornará el código en el lenguaje objetivo.

\begin{figure}[H]
\centering
\resizebox{10cm}{!} {
\includegraphics{img/etapas_compilador.png}
}
\caption{Etapas del desarrollo de un compilador, indicando los dos procesos principales.}
\label{fig:etapas}
\end{figure}

El desarrollo tiene dos procesos principales, el \emph{Análisis} y la \emph{Síntesis}. El análisis es el proceso en que se dará énfasis, ya que consta de las etapas más importantes como el análisis léxico, sintáctico y semántico, teniendo principal hincapié en la sintaxis ya que se realizará una \emph{traducción dirigida por la sintaxis} para este trabajo.

En este sentido, para este informe vamos a mostrar el diseño y la implementación de un compilador para un conjunto reducido de construcciones del lenguaje Pascal, explicando el desarrollo de cada etapa y cómo se relaciona con la etapa siguiente.

%\end{adjustwidth*}
\setcounter{page}{1}

\input{gramatica}

\chapter{Especificación léxica}

\section{Introducción}
En este capítulo utilizaremos la gramática desarrollada en el capítulo anterior y realizaremos el diseño de un analizador léxico. El objetivo es recorrer la cadena de entrada carácter por carácter y detectar los {\bf tokens} especificados, en este caso Pascal reducido (definido por la gramática).
Para ello, se indicarán como están formados los \emph{tokens}, que son elementos que utilizará el analizador sintáctico posteriormente, y se mostrará el diseño de un autómata que permita el reconocimiento de tales tokens en un recorrido [\ref{sec:lexico_reconocedor}].

\section{Descripción del problema}
Para esta etapa requerimos especificar en un principio el alfabeto que tendrá nuestra cadena de entrada, luego debemos definir que combinación de símbolos corresponderán a lexemas de nuestro lenguaje y partir de éstos debemos generar patrones que identifiquen para cada token que lexemas les corresponden [\ref{sec:lexico_definicion}].

Dados los tokens y sus patrones representativos, se especifica un autómata finito determinístico que lee cada símbolo de la cadena de entrada y permite determinar si es un token válido del lenguaje. Para ello el autómata se realiza mediante la unión de autómatas más simples para reconocer los patrones de cada token por separado.

\section{Definición del alfabeto, lexemas, tokens y patrones}
\label{sec:lexico_definicion}
El alfabeto estará compuesto por las letras del abecedario (A-Z y a-z) en conjunción con los dígitos de 0-9 y algunos caracteres especiales:

$$ \Sigma = \{ \bm{A}, ..., \bm{Z}, \bm{a}, ..., \bm{z}, \bm{0}, ..., \bm{9}, \bm{\_}, \bm{\}}, \bm{\{}, \bm{)}, \bm{(}, \bm{;}, \bm{\cdot}, \bm{,}, \bm{:}, \bm{=}, \bm{<}, \bm{>}, \bm{+}, \bm{-}, \bm{*}, \bm{/} \}$$

Con estos símbolos podemos definir los lexemas que componen el subconjunto de lenguaje Pascal que se desarrolla y luego definir expresiones regulares para poder identificar a que token corresponden.

En la tabla \ref{tab:tabla_token} podemos visualizar los tokens que se identifican en el lenguaje especificado junto con los patrones que definen los lexemas que se corresponden con cada uno de estos tokens. Tales patrones se especificarán mediante expresiones regulares.

\begin{table}[H]
\centering
\begin{tabular}{|l|l|l|}
\rowcolor{gray!20}
\hline
Token         & Patrón                                               & Ejemplo              \\ \hline
tk\_type\_int      & $integer$                                  & $integer$            \\ \hline
tk\_type\_bool      & $boolean$                                  & $boolean$            \\ \hline
tk\_boolean\_true   & $true$                                       & $true$               \\ \hline
tk\_boolean\_false   & $false$                                       & $false$               \\ \hline
tk\_number   & $[0-9]^+$                                            & $100$                \\ \hline
tk\_id        & $([A-Z] | [a-z] | \_)([A-Z] | [a-z] | \_ | [0-9])^*$ & $un\_identificador\_1$ \\ \hline
tk\_assign    & $:=$                                                 & $:=$                 \\ \hline
tk\_rel\_op\_eq   & $=$                           & $=$                 \\ \hline
tk\_rel\_op\_neq   & $<>$                           & $<>$                 \\ \hline
tk\_rel\_op\_min   & $<$                           & $<$                 \\ \hline
tk\_rel\_op\_may   & $>$                           & $>$                 \\ \hline
tk\_rel\_op\_leq   & $<=$                           & $<=$                 \\ \hline
tk\_rel\_op\_geq   & $>=$                           & $>=$                 \\ \hline
tk\_add\_op\_sum   & $+$                                             & $+$                  \\ \hline
tk\_add\_op\_rest   & $-$                                             & $-$                  \\ \hline
tk\_mult\_op\_por  & $*$                                              & $*$                  \\ \hline
tk\_mult\_op\_div  & $/$                                              & $/$                  \\ \hline
tk\_bool\_op\_and  & $and$                                           & $and$                \\ \hline
tk\_bool\_op\_or  & $or$                                           & $or$                \\ \hline
tk\_not\_op   & $not$                                                & $not$                \\ \hline
tk\_if        & $if$                                                 & $if$                 \\ \hline
tk\_then      & $then$                                               & $then$               \\ \hline
tk\_else      & $else$                                               & $else$               \\ \hline
tk\_while     & $while$                                              & $while$              \\ \hline
tk\_do        & $do$                                                 & $do$                 \\ \hline
tk\_program   & $program$                                            & $program$            \\ \hline
tk\_begin     & $begin$                                              & $begin$              \\ \hline
tk\_end       & $end$                                                & $end$                \\ \hline
tk\_var       & $var$                                                & $var$                \\ \hline
tk\_procedure & $procedure$                                          & $procedure$          \\ \hline
tk\_function  & $function$                                           & $function$           \\ \hline
tk\_read  & $read$                                           & $read$           \\ \hline
tk\_write  & $write$                                           & $write$           \\ \hline
tk\_opar      & (                                                    & (                    \\ \hline
tk\_cpar      & )                                                    & )                    \\ \hline
tk\_tpoints   & :                                                    & :                    \\ \hline
tk\_endstnc   & ;                                                    & ;                    \\ \hline
tk\_point     & .                                                    & .                    \\ \hline
tk\_comma     & ,                                                    & ,                    \\ \hline
\end{tabular}
\caption{Define para cada token el patrón que identifica los lexemas que le corresponden.}
\label{tab:tabla_token}
\end{table}

\section{Estrategias}
Utilizamos diferentes tokens para cada uno de los operadores, como en los operadores de comparación, de multiplicación, y de suma. Así aunque se manejen más cantidad de tokens que si estuvieren agrupados, es más sencillo diferenciarlos en las etapas posteriores.

Para componer el reconocedor, tomaremos cualquier cadena de entrada que pueda ser una palabra clave, con el patrón del token tk\_id. Si es una palabra clave, se crea su token correspondiente, si no lo es, entonces se confirma que es un identificador. De esta manera se simplifica el autómata reconocedor de tokens.

\section{Limitaciones}
\label{sec:lexico_limitaciones}
En la especificación de tokens se ha indicado para cada operador un token diferente, como con los operadores relacionales (rel\_op), esto permite diferenciarlos y en futuras etapas facilita la implementación. En cambio se pudo optar por una visión más genérica que facilite la visualización de los diferentes token conceptualmente, de manera que queden agrupados y se diferencien por un atributo que indique específicamente que lexema representa.

Por otro lado, con respecto al diseño del autómata, se consideró, para simplificar el diseño, que las palabras reservadas no serán reconocidas por estados independientes si no que serán un caso específico del patrón para reconocer identificadores (tk\_id).

\section{Diseño del reconocedor}
\label{sec:lexico_reconocedor}

% \begin{figure}[H]
% \begin{center}
% \begin{tikzpicture}[scale=0.15]
% \tikzstyle{every node}+=[inner sep=0pt]
% \draw [black] (49,-15.4) circle (3);
% \draw (49,-15.4) node {$2$};
% \draw [black] (49,-2.8) circle (3);
% \draw (49,-2.8) node {$1$};
% \draw [black] (64.8,-15.4) circle (3);
% \draw (64.8,-15.4) node {$3\mbox{ }*$};
% \draw (76,-15.4) node {{\bf return($tk\_id$)}};
% \draw [black] (64.8,-15.4) circle (2.4);
% \draw [black] (22.5,-2.8) circle (3);
% \draw (22.5,-2.8) node {$start$};
% \draw [black] (49,-28.4) circle (3);
% \draw (49,-28.4) node {$4$};
% \draw [black] (65.4,-28.4) circle (3);
% \draw (65.4,-28.4) node {$5\mbox{ }*$};
% \draw (78.5,-28.4) node {{\bf return($tk\_integer$)}};
% \draw [black] (65.4,-28.4) circle (2.4);
% \draw [black] (49,-38.3) circle (3);
% \draw (49,-38.3) node {$6$};
% \draw [black] (65.4,-42.6) circle (3);
% \draw (65.4,-42.6) node {$7$};
% \draw (78.5,-42.6) node {{\bf return($tk\_assign$)}};
% \draw [black] (65.4,-42.6) circle (2.4);
% \draw [black] (48.5,-47.4) circle (3);
% \draw (48.5,-47.4) node {$8$};
% \draw (63.4,-46.9) node {{\bf return($tk\_rel\_op\_eq$)}};
% \draw [black] (48.5,-47.4) circle (2.4);
% \draw [black] (48.5,-56.2) circle (3);
% \draw (48.5,-56.2) node {$9$};
% \draw [black] (65.4,-51.7) circle (3);
% \draw (65.4,-51.7) node {$10\mbox{ }*$};
% \draw (80,-51.7) node {{\bf return($tk\_rel\_op\_min$)}};
% \draw [black] (65.4,-51.7) circle (2.4);
% \draw [black] (65.4,-59.1) circle (3);
% \draw (65.4,-59.1) node {$11$};
% \draw (80,-59.1) node {{\bf return($tk\_rel\_op\_leq$)}};
% \draw [black] (65.4,-59.1) circle (2.4);
% \draw [black] (65.4,-66) circle (3);
% \draw (65.4,-66) node {$12$};
% \draw (80,-66) node {{\bf return($tk\_rel\_op\_neq$)}};
% \draw [black] (65.4,-66) circle (2.4);
% \draw [black] (48.5,-72.8) circle (3);
% \draw (48.5,-72.8) node {$13$};
% \draw [black] (65.4,-72.8) circle (3);
% \draw (65.4,-72.8) node {$14$};
% \draw (80,-72.8) node {{\bf return($tk\_rel\_op\_geq$)}};
% \draw [black] (65.4,-72.8) circle (2.4);
% \draw [black] (65.4,-80.7) circle (3);
% \draw (65.4,-80.7) node {$15\mbox{ }*$};
% \draw (80,-80.7) node {{\bf return($tk\_rel\_op\_may$)}};
% \draw [black] (65.4,-80.7) circle (2.4);
% \draw [black] (49,-86.4) circle (3);
% \draw (49,-86.4) node {$16$};
% \draw (63,-86.4) node {{\bf return($tk\_add\_op$)}};
% \draw [black] (49,-86.4) circle (2.4);
% \draw [black] (49,-95.6) circle (3);
% \draw (49,-95.6) node {$18$};
% \draw (63,-95.6) node {{\bf return($tk\_mult\_op$)}};
% \draw [black] (49,-95.6) circle (2.4);
% \draw [black] (65.4,-35.7) circle (3);
% \draw (65.4,-35.7) node {$19\mbox{ }*$};
% \draw (78.5,-35.7) node {{\bf return($tk\_tpoins$)}};
% \draw [black] (65.4,-35.7) circle (2.4);
% \draw [black] (51.253,-0.837) arc (158.78564:-129.21436:2.25);
% \draw (56.29,-0.58) node [right] {$otro$};
% \fill [black] (51.93,-3.39) -- (52.49,-4.15) -- (52.86,-3.22);
% \draw [black] (47.677,-12.72) arc (234:-54:2.25);
% \draw (49,-8.15) node [above] {$letter\mbox{ }or\mbox{ }digit\mbox{ }or\mbox{ }\_$};
% \fill [black] (50.32,-12.72) -- (51.2,-12.37) -- (50.39,-11.78);
% \draw [black] (52,-15.4) -- (61.8,-15.4);
% \fill [black] (61.8,-15.4) -- (61,-14.9) -- (61,-15.9);
% \draw (56.9,-14.9) node [above] {$otro$};
% \draw [black] (46,-2.8) -- (25.5,-2.8);
% \fill [black] (25.5,-2.8) -- (26.3,-3.3) -- (26.3,-2.3);
% \draw (35.75,-3.3) node [below] {$\}$};
% \draw [black] (25.231,-1.562) arc (111.4063:68.5937:28.821);
% \fill [black] (46.27,-1.56) -- (45.71,-0.8) -- (45.34,-1.74);
% \draw (35.75,0.93) node [above] {$\{$};
% \draw [black] (25.21,-4.09) -- (46.29,-14.11);
% \fill [black] (46.29,-14.11) -- (45.78,-13.32) -- (45.35,-14.22);
% \draw (31.15,-9.63) node [below] {$letter\mbox{ }or\mbox{ }\_$};
% \draw [black] (46.059,-27.813) arc (-103.99149:-164.0292:31.782);
% \fill [black] (46.06,-27.81) -- (45.4,-27.13) -- (45.16,-28.1);
% \draw (29.31,-20.31) node [below] {$digit$};
% \draw [black] (47.677,-25.72) arc (234:-54:2.25);
% \draw (49,-21.15) node [above] {$digit$};
% \fill [black] (50.32,-25.72) -- (51.2,-25.37) -- (50.39,-24.78);
% \draw [black] (52,-28.4) -- (62.4,-28.4);
% \fill [black] (62.4,-28.4) -- (61.6,-27.9) -- (61.6,-28.9);
% \draw (57.2,-28.9) node [below] {$otro$};
% \draw [black] (46.002,-38.324) arc (-92.79513:-193.72375:26.42);
% \fill [black] (46,-38.32) -- (45.23,-37.79) -- (45.18,-38.78);
% \draw (25.54,-29.13) node [left] {$:$};
% \draw [black] (51.9,-39.06) -- (62.5,-41.84);
% \fill [black] (62.5,-41.84) -- (61.85,-41.15) -- (61.6,-42.12);
% \draw (58,-39.88) node [above] {$=$};
% \draw [black] (45.501,-48.422) arc (-92.52263:-208.09592:29.287);
% \fill [black] (45.5,-48.42) -- (44.72,-47.89) -- (44.68,-48.89);
% \draw (20.69,-34.89) node [left] {$=$};
% \draw [black] (45.517,-56.504) arc (-86.95094:-221.12695:31.119);
% \fill [black] (45.52,-56.5) -- (44.69,-56.05) -- (44.74,-57.05);
% \draw (15.18,-40.13) node [left] {$<$};
% \draw [black] (51.4,-55.43) -- (62.5,-52.47);
% \fill [black] (62.5,-52.47) -- (61.6,-52.19) -- (61.86,-53.16);
% \draw (55.34,-53.32) node [above] {$otro$};
% \draw [black] (51.46,-56.71) -- (62.44,-58.59);
% \fill [black] (62.44,-58.59) -- (61.74,-57.96) -- (61.57,-58.95);
% \draw (56.46,-58.25) node [below] {$=$};
% \draw [black] (51.1,-57.7) -- (62.8,-64.5);
% \fill [black] (62.8,-64.5) -- (62.36,-63.66) -- (61.86,-64.53);
% \draw (55.89,-61.6) node [below] {$>$};
% \draw [black] (45.516,-72.502) arc (-97.80406:-221.44306:40.861);
% \fill [black] (45.52,-72.5) -- (44.79,-71.9) -- (44.66,-72.89);
% \draw (12,-47.06) node [left] {$>$};
% \draw [black] (51.5,-72.8) -- (62.4,-72.8);
% \fill [black] (62.4,-72.8) -- (61.6,-72.3) -- (61.6,-73.3);
% \draw (56.95,-73.3) node [below] {$=$};
% \draw [black] (51.22,-74.07) -- (62.68,-79.43);
% \fill [black] (62.68,-79.43) -- (62.17,-78.64) -- (61.75,-79.54);
% \draw (54.87,-77.27) node [below] {$otro$};
% \draw [black] (46.068,-85.769) arc (-103.8764:-220.94786:49.664);
% \fill [black] (46.07,-85.77) -- (45.41,-85.09) -- (45.17,-86.06);
% \draw (9.87,-53.22) node [left] {$+\mbox{ }or\mbox{ }-$};
% \draw [black] (46.027,-95.204) arc (-99.23615:-228.88937:51.996);
% \fill [black] (46.03,-95.2) -- (45.32,-94.58) -- (45.16,-95.57);
% \draw (3.61,-58.72) node [left] {$*\mbox{ }or\mbox{ }/$};
% \draw [black] (19.572,-3.399) arc (309.29645:21.29645:2.25);
% \draw (15.21,-0.59) node [left] {$ws$};
% \fill [black] (20.24,-0.84) -- (20.12,0.1) -- (19.35,-0.54);
% \draw [black] (51.96,-37.83) -- (62.44,-36.17);
% \fill [black] (62.44,-36.17) -- (61.57,-35.8) -- (61.73,-36.79);
% \draw (56.34,-36.3) node [above] {$otro$};
% \end{tikzpicture}
% \end{center}
% \caption{Autómata reconocedor de tokens.}
% \label{fig:automata_token2}
% \end{figure}

\begin{figure}[H]
\centering
\begin{tikzpicture}[scale=0.65, node distance = 2.5cm, state/.style={circle, draw, minimum size=1cm}]
\tikzset{every state/.append style={thick, fill=gray!10}}
\node[state] (0) at (0,0) {$start$};
\node[state] (1) at (0,-6) {$1$};
\node[state] (2) at (3,6) {$2$};
\node[state, accepting] (3) at (6,6) {$3$ $*$};
\node[state] (4) at (3,2) {$4$};
\node[state, accepting] (5) at (6,4) {$5$ $*$};
\node[state] (6) at (3,0) {$6$};
\node[state, accepting] (7) at (6,2) {$7$ $*$};
\node[state, accepting] (8) at (6,0) {$8$};
\node[state] (9) at (3,-4) {$9$};
\node[state, accepting] (10) at (6,-2) {$10$ $*$};
\node[state, accepting] (11) at (6,-4) {$11$};
\node[state, accepting] (12) at (6,-6) {$12$};
\node[state] (13) at (-3,5) {$13$};
\node[state, accepting] (14) at (-6,6) {$14$ $*$};
\node[state, accepting] (15) at (-6,4) {$15$};
\node[state, accepting] (16) at (-6,2) {$16$};
\node[state, accepting] (17) at (-6,0) {$17$};
\node[state, accepting] (18) at (-6,-2) {$18$};
\node[state, accepting] (19) at (-6,-4) {$19$};
\node[state, accepting] (20) at (-6,-6) {$20$};

\node[right] at (7,6) {{\bf return($tk\_id$)}};
\node[right] at (7,4) {{\bf return($tk\_number$)}};
\node[right] at (7,2) {{\bf return($tk\_tpoints$)}};
\node[right] at (7,0) {{\bf return($tk\_assign$)}};
\node[right] at (7,-2) {{\bf return($tk\_rel\_op\_min$)}};
\node[right] at (7,-4) {{\bf return($tk\_rel\_op\_leq$)}};
\node[right] at (7,-6) {{\bf return($tk\_rel\_op\_neq$)}};

\node[left] at (-7,6) {{\bf return($tk\_rel\_op\_may$)}};
\node[left] at (-7,4) {{\bf return($tk\_rel\_op\_geq$)}};
\node[left] at (-7,2) {{\bf return($tk\_rel\_op\_eq$)}};
\node[left] at (-7,0) {{\bf return($tk\_add\_op\_sum$)}};
\node[left] at (-7,-2) {{\bf return($tk\_add\_op\_rest$)}};
\node[left] at (-7,-4) {{\bf return($tk\_mult\_op\_por$)}};
\node[left] at (-7,-6) {{\bf return($tk\_mult\_op\_div$)}};

\path[->,>=stealth]
(0) edge[loop above] node{$ws$} (0)	
	edge[bend left, left] node{$\{$} (1)
    edge[above] node{$letter$ or $\_$} (2)
    edge[above] node{$digit$} (4)
    edge[above] node{$:$} (6)
    edge[above] node{$<$} (9)
    edge[left] node{$>$} (13)
    edge[above] node{$=$} (16)
    edge[above] node{$+$} (17)
    edge[above] node{$-$} (18)
    edge[above] node{$*$} (19)
    edge[above] node{$/$} (20)
(1) edge[loop below] node{$otro$} (1)
	edge[bend left, right] node{$\}$} (0)
(2) edge[loop above] node{$letter$ or $digit$ or $\_$} (2)
	edge[above] node{$otro$} (3)
(4) edge[loop above] node{$digit$} (4)
	edge[above] node{$otro$} (5)
(6) edge[above] node{$otro$} (7)
	edge[above] node{$=$} (8)
(9) edge[above] node{$otro$} (10)
	edge[above] node{$=$} (11)
	edge[above] node{$>$} (12)
(13) edge[above] node{$otro$} (14)
	 edge[above] node{$=$} (15)
;
\end{tikzpicture}
\caption{Autómata reconocedor de tokens.}
\label{fig:automata_token}
\end{figure}

Podemos contemplar que en ciertos estados se consume un elemento \emph{otro}, el cuál dependerá de tal estado, y simboliza un elemento que no corresponde con los símbolos cuyos lexemas están incluidos en el patrón del token que está reconociendo. Por ejemplo, en el arco del estado 2-3, \emph{otro} corresponde a símbolos que no son \emph{letter}, \emph{digit} o \emph{\_}. En el caso del arco 4-5 corresponde un símbolo que no es \emph{digit}. En 9-10 para cualquier símbolo que no es \emph{=} o \emph{$>$} y en 13-15 para cualquiera que no sea \emph{=}. Finalmente en el arco del estado 1, se utiliza de manera que reconozca cualquier símbolo, ya que este es el encargado de reconocer las secciones comentadas del código, lo que permite ignorar cualquier símbolo encerrado entre llaves.

Por otro lado, encontramos el arco etiquetado con \emph{ws} que permite ignorar los espacios, los saltos de línea y las tabulaciones.

A modo de simplificación del gráfico, tanto por cuestiones de tamaño como de estética, no se indicó el reconocimiento de los siguientes tokens: tk\_opar, tk\_cpar, tk\_tpoints, tk\_endstnc, tk\_point, tk\_comma. Estos patrones son triviales, y serían análogos, por ejemplo, al reconocimiento de tk\_mult\_op.
%\section{Problemas encontrados}

\section{Implementación del aplicativo}
\label{sec:implementacion_lexico}
En esta sección proponemos una implementación de un analizador léxico que se corresponde con la especificación léxica de la sección anterior.

Se dispondrá el enlace del código fuente alojado en GitHub\footnote{\url{https://github.com/Martinnqn/CompiladorPascal}}. Durante el desarrollo de este trabajo se ampliará el aplicativo con las demás etapas de análisis sintáctico y semántico. De esta manera, esperamos concluir todas las etapas del compilador para el lenguaje elegido.  

\subsection{Descripción del problema}
La idea es desarrollar y documentar un programa capaz de reconocer lexemas de un código fuente basado en la gramática de la sección \ref{sec:definicion_gramatica}, y devolver sus tokens asociados en la tabla \ref{tab:tabla_token}. Para esto, usaremos el autómata de la figura \ref{fig:automata_token}.

\subsection{Herramientas utilizadas} %ver titulo..
Para desarrollar el programa escogimos el lenguaje Java. La versión de Java sobre la que trabajamos es la 8\footnote{Actualización 171, al día 14/05/2018}(ocho), sobre el sistema operativo Windows 10. 

El código de fuente alojado en GitHub tiene la estructura de un proyecto de NetBeans, por lo que puede usarse ese IDE para levantar el proyecto.

\subsection{Diseño}
Para llevar a cabo lo propuesto en este capítulo, y luego poder continuar con el mismo proyecto ampliando el código con las siguientes etapas de análisis, propusimos estructurar el proyecto en \emph{packages}. La estructura del proyecto se puede ver en la figura \ref{fig:arbol_dir}, por lo que ahora nos centraremos en trabajar sobre el paquete \emph{lexico}, el cual contiene todas las clases que se usarán en el análisis léxico.

\begin{figure}[H]
%no borrar el % de dirtree porque es necesario.
\dirtree{%
.1 src.
.2 compiladorpascal.
.3 CompiladorPascal.java.
.3 lexico.
.3 sintactico.
.3 semantico.
}
\caption{Árbol de directorios del proyecto de Java.}
\label{fig:arbol_dir}
\end{figure}
La clase {\bf CompiladorPascal} será la clase principal del proyecto, donde se crearan los objetos necesarios en cada etapa del análisis. 

Las clases que se usarán en el análisis léxico son: {\bf AnalizadorLexico, Token, Tokens}. El árbol de directorios actualizado queda como el de la figura \ref{fig:arbol_dir_2}:
\begin{figure}[H]
%no borrar el % de dirtree porque es necesario.
\dirtree{%
.1 src.
.2 compiladorpascal.
.3 CompiladorPascal.java.
.3 lexico.
.4 AnalizadorLexico.java.
.4 Token.java.
.4 Tokens.java.
.3 sintactico.
.3 semantico.
}
\caption{Árbol de directorios del proyecto de Java con los archivos del analizador léxico.}
\label{fig:arbol_dir_2}
\end{figure}

{\bf Descripción de las clases}: 
\begin{itemize}
\item {\bf AnalizadorLexico}: contiene el código necesario para reconocer cada token y reportar los errores léxicos en caso de haberlos. El código tiene una relación directa con el autómata de la figura \ref{fig:automata_token}.
%\item {\bf ErrorLéxico}: contiene atributos necesarios para mostrar mensajes detallados acerca de los errores léxicos que ocurran durante la compilación. Para cada error se sugiere un mensaje apropiado que permita al programador encontrar el error en el código y poder solucionarlo. 
\item {\bf Token}: contiene los atributos que representan un token, tal como su nombre y valor.
\item {\bf Tokens}: contiene métodos estáticos que almacenan en HashMaps los nombres y valores de los tokens como palabras reservadas y símbolos. Se utilizará en el AnalizadorLexico para obtener los nombres y patrones de los token del lenguaje.
\end{itemize}

%\subsubsection{Estrategias}

%\subsubsection{Limitaciones}

%\subsubsection{Problemas encontrados}

\subsection{Instructivos de instalación y uso}
Al ser un programa hecho en Java, cuenta con la capacidad de ser portable, por lo que no requiere instalación.

El proceso de compilación puede realizarse mediante un IDE (NetBeans en nuestro caso), o mediante la consola o terminal, con el comando \emph{javac *.java} sobre los archivos del paquete.   

\subsection{Ejemplos}
En la figura \ref{fig:lexico_ej_correcto} vemos un programa en nuestro Pascal que es léxicamente correcto, por lo que la salida en la figura \ref{fig:lexico_ej_correcto_salida} nos muestra una cadena con los tokens obtenidos tras el análisis y no menciona ningún error encontrado. Por otro lado en las figura \ref{fig:lexico_ej_error_1} encontramos un código con un caracter desconocido ``$\%$'' y en la figura \ref{fig:lexico_ej_error_1_salida} la correspondiente salida con el error, indicando la línea y posición donde ocurrió y más abajo los tokens que reconoció hasta alcanzar el error. También podemos ver en la figura \ref{fig:lexico_ej_error_2} un código donde no se cierra el comentario por lo que en la figura \ref{fig:lexico_ej_error_2_salida} vemos que la salida es un error específico para este caso.

\begin{figure}[H]
\begin{minted}[autogobble,linenos,xleftmargin=0.35\textwidth,xrightmargin=0.35\textwidth]{pascal}
Program Example1;
Var       
    Num1, Num2, Sum : Integer;
    Result: Boolean;
Begin {no semicolon}
    Sum := Num1 + Num2;
    if (Num1 > Num2) then
        Result := true
End.
\end{minted}
\caption{Programa en Pascal reducido léxicamente correcto.}
\label{fig:lexico_ej_correcto}
\end{figure}

\begin{figure}[H]
\centering
\includegraphics[scale=1]{img/lexico/salida_lexico_ej_correcto.png}
\caption{Salida de la ejecución del analizador léxico con el código de la figura \ref{fig:lexico_ej_correcto}.}
\label{fig:lexico_ej_correcto_salida}
\end{figure}

\begin{figure}[H]
\begin{minted}[autogobble,linenos,xleftmargin=0.35\textwidth,xrightmargin=0.35\textwidth]{pascal}
Program Example2;
Var       
    %Num1, Num2, Sum : Integer;
    Result: Boolean;
Begin [ {no semicolon}
    Sum := Num1 ++ Num2;
    if (Num1 > Num2) then
        Result := true
]
End.
\end{minted}
\caption{Programa en Pascal con error léxico por caracter desconocido.}
\label{fig:lexico_ej_error_1}
\end{figure}

\begin{figure}[H]
\centering
\includegraphics[scale=1]{img/lexico/salida_lexico_ej_error_1.png}
\caption{Salida de la ejecución del analizador léxico con el código de la figura \ref{fig:lexico_ej_error_1}.}
\label{fig:lexico_ej_error_1_salida}
\end{figure}

\begin{figure}[H]
\begin{minted}[autogobble,linenos,xleftmargin=0.35\textwidth,xrightmargin=0.35\textwidth]{pascal}
Program Example3;
Var       
    Num1, Num2, Sum : Integer;
    Result: Boolean;
Begin {no semicolon}
    {Sum := Num1 + Num2;
    if (Num1 > Num2) then
        Result := true
End.
\end{minted}
\caption{Programa en Pascal con error léxico por no encontrar final de comentario.}
\label{fig:lexico_ej_error_2}
\end{figure}

\begin{figure}[H]
\centering
\includegraphics[scale=1]{img/lexico/salida_lexico_ej_error_2.png}
\caption{Salida de la ejecución del analizador léxico con el código de la figura \ref{fig:lexico_ej_error_2}.}
\label{fig:lexico_ej_error_2_salida}
\end{figure}

\section{Conclusiones}
%Realizamos una especificación del autómata de una manera simplificada de manera que sea más visible su funcionamiento [\ref{sec:lexico_reconocedor}] y a futuro se dará más detalle a la hora de implementar. De la misma manera para la definición de los token en la tabla \ref{tab:tabla_token} se consideró cada token de una manera más conceptual y por ello no se especificó uno separado para cada lexema, como se menciona en \ref{sec:lexico_limitaciones}. En este sentido la especificación dada en este capítulo tiene una visión más general del funcionamiento de un analizador léxico y no tan allegada a la implementación.
En este capítulo pudimos concluir la especificación e implementación del análisis léxico. Realizamos una definición de los tokens en la tabla \ref{tab:tabla_token}, considerando cada token conceptualmente como se menciona en \ref{sec:lexico_limitaciones} y así evitamos una definición exhaustiva y simplificamos la tabla. 

En base a la tabla de tokens, diseñamos el autómata reconocedor de tokens [\ref{sec:lexico_reconocedor}] describiendo su funcionamiento. Luego, en la sección \ref{sec:implementacion_lexico}, tuvimos en cuenta las especificaciones y desarrollamos un programa en Java que permite reconocer los tokens de nuestro lenguaje.

En los siguientes capítulos continuaremos desarrollando las especificaciones e implementaciones de las siguientes etapas de un compilador.

\input{sintactico}

\input{semantico}

\input{generacion_codigo}

\chapter{Conclusiones}
%las conclusiones de las conclusiones

\section{Trabajo realizado}
En este trabajo concluimos con el diseño y la implementación de un compilador para un subconjunto del lenguaje Pascal. Se han atravesado varias etapas desde el diseño inicial de la gramática que reconoce el lenguaje de Pascal reducido, seguido de los analizadores léxico, sintáctico y semántico, y finalmente concluyendo con el generador de código intermedio para utilizar sobre la máquina virtual MEPa.

En cada etapa tomamos las decisiones adecuadas para abordar los problemas que se presentaron. Como el tratamiento a la ambigüedad del lenguaje en el análisis sintáctico, las estructuras elegidas e implementadas para representar la tabla de símbolos en el análisis semántico y finalmente aprovechar las llamadas recursivas propias de la implementación sintáctica, dada por la precedencia de los operadores, para la generación de código intermedio, como también el uso de la tabla de símbolos definida previamente.

\section{Mejoras y futuros trabajos}
Como mejoras aún quedan chequeos semánticos por realizar, como verificar que las variables estén inicializadas antes de utilizarlas en operaciones. Además, queda pendiente la optimización de código intermedio antes de enviarlo a la máquina virtual. Por otro lado tampoco se realizó la generación de código objeto, ya que fue delegada a la máquina virtual MEPa.

\chapter*{Apéndice}
\appendix
%\markboth{Appendices}{}
\addcontentsline{toc}{chapter}{Apéndice}
\renewcommand{\thesection}{A.\arabic{section}}

\section{Pseudocódigo Analizador Sintáctico}
\label{apx:pseudocodigo_sintactico}

\begin{multicols}{2}
\raggedcolumns
\interlinepenalty=10000
\begin{minted}[autogobble,fontsize=\small,xleftmargin=1cm,tabsize=1]{java}
void match(terminal) {
	if (preanalisis == terminal) {
		preanalisis = tokenSiguiente();
	} else {
		error();
	}
}
\end{minted}
\begin{minted}[autogobble,fontsize=\small,xleftmargin=1cm,tabsize=1]{java}
void program() {
	if (preanalisis == TK_PROGRAM) {
		program_heading();
		block();
		match(TK_POINT);
	} else {
		error();
	}
}
\end{minted}
\begin{minted}[autogobble,fontsize=\small,xleftmargin=1cm,tabsize=1]{java}
void program_heading() {
	if (preanalisis == TK_PROGRAM) {
		match(TK_PROGRAM);
		identifier();
		match(TK_ENDSTNC);
	} else {
		error();
	}
}
\end{minted}
\begin{minted}[autogobble,fontsize=\small,xleftmargin=1cm,tabsize=1]{java}
void block() {
	switch (preanalisis) {
		case TK_VAR:
		case TK_PROCEDURE:
		case TK_FUNCTION:
			declaration_block();
			multiple_statement();
			break;
		case TK_BEGIN:
			multiple_statement();
			break;
		default:
			error();
			break;
	}
}
\end{minted}
\begin{minted}[autogobble,fontsize=\small,xleftmargin=1cm,tabsize=1]{java}
void declaration_block() {
	switch (preanalisis) {
		case TK_VAR:
			variable_declaration_block();
			declaration_block_1();
			break;
		case TK_PROCEDURE:
		case TK_FUNCTION:
			declaration_block_1();
			break;
		default:
			error();
			break;
	}
}
\end{minted}
\begin{minted}[autogobble,fontsize=\small,xleftmargin=1cm,tabsize=1]{java}
void declaration_block_1() {
	switch (preanalisis) {
		case TK_PROCEDURE:
		case TK_FUNCTION:
			procedure_and_function_declaration_list();
			break;
	}
}
\end{minted}
\begin{minted}[autogobble,fontsize=\small,xleftmargin=1cm,tabsize=1]{java}
void variable_declaration_block() {
	if (preanalisis == TK_VAR)) {
		match(TK_VAR);
		variable_declaration_list();
	} else {
		error();
	}
}
\end{minted}
\begin{minted}[autogobble,fontsize=\small,xleftmargin=1cm,tabsize=1]{java}
void variable_declaration_list() {
	if (preanalisis == TK_ID)) {
		variable_declaration();
		match(TK_ENDSTNC);
		variable_declaration_list_1();
	} else {
		error();
	}
}
\end{minted}
\begin{minted}[autogobble,fontsize=\small,xleftmargin=1cm,tabsize=1]{java}
void variable_declaration_list_1() {
	if (preanalisis == TK_ID)) {
		variable_declaration_list();
	}
}
\end{minted}
\begin{minted}[autogobble,fontsize=\small,xleftmargin=1cm,tabsize=1]{java}
void variable_declaration() {
	if (preanalisis == TK_ID)) {
		identifier_list();
		match(TK_TPOINTS);
		type();
	} else {
		error();
	}
}
\end{minted}
\begin{minted}[autogobble,fontsize=\small,xleftmargin=1cm,tabsize=1]{java}
void procedure_and_function_declaration_list() {
	switch (preanalisis) {
		case TK_PROCEDURE:
			procedure_declaration();
			match(TK_ENDSTNC);
			procedure_and_function_declaration_list_1();
			break;
		case TK_FUNCTION:
			function_declaration();
			match(TK_ENDSTNC);
			procedure_and_function_declaration_list_1();
			break;
		default:
			error();
			break;
	}
}
\end{minted}
\begin{minted}[autogobble,fontsize=\small,xleftmargin=1cm,tabsize=1]{java}
void procedure_and_function_declaration_list_1() {
	switch (preanalisis) {
		case TK_PROCEDURE:
		case TK_FUNCTION:
			procedure_and_function_declaration_list();
			break;
	}
}
\end{minted}
\begin{minted}[autogobble,fontsize=\small,xleftmargin=1cm,tabsize=1]{java}
void procedure_declaration() {
	if (preanalisis == TK_PROCEDURE)) {
		procedure_heading();
		match(TK_ENDSTNC);
		block();
	} else {
		error();
	}
}
\end{minted}
\begin{minted}[autogobble,fontsize=\small,xleftmargin=1cm,tabsize=1]{java}
void procedure_heading() {
	if (preanalisis == TK_PROCEDURE)) {
		match(TK_PROCEDURE);
		identifier();
		parameters();
	} else {
		error();
	}
}
\end{minted}
\begin{minted}[autogobble,fontsize=\small,xleftmargin=1cm,tabsize=1]{java}
void function_declaration() {
	if (preanalisis == TK_FUNCTION)) {
		function_heading();
		match(TK_ENDSTNC);
		block();
	} else {
		error();
	}
}
\end{minted}
\begin{minted}[autogobble,fontsize=\small,xleftmargin=1cm,tabsize=1]{java}
void function_heading() {
	if (preanalisis == TK_FUNCTION)) {
		match(TK_FUNCTION);
		identifier();
		parameters();
		match(TK_TPOINTS);
		type();
	} else {
		error();
	}
}
\end{minted}
\begin{minted}[autogobble,fontsize=\small,xleftmargin=1cm,tabsize=1]{java}
void parameters() {
	if (preanalisis == TK_OPAR)) {
		match(TK_OPAR);
		parameters_2();
		match(TK_CPAR);
	}
}
\end{minted}
\begin{minted}[autogobble,fontsize=\small,xleftmargin=1cm,tabsize=1]{java}
void parameters_2() {
	if (preanalisis == TK_ID)) {
		parameter_declaration_list();
	}
}
\end{minted}
\begin{minted}[autogobble,fontsize=\small,xleftmargin=1cm,tabsize=1]{java}
void parameter_declaration_list() {
	if (preanalisis == TK_ID)) {
		parameter_declaration();
		parameter_declaration_list_1();
	} else {
		error();
	}
}
\end{minted}
\begin{minted}[autogobble,fontsize=\small,xleftmargin=1cm,tabsize=1]{java}
void parameter_declaration_list_1() {
	if (preanalisis == TK_COMMA)) {
		match(TK_COMMA);
		parameter_declaration_list();
	}
}
\end{minted}
\begin{minted}[autogobble,fontsize=\small,xleftmargin=1cm,tabsize=1]{java}
void parameter_declaration() {
	if (preanalisis == TK_ID)) {
		identifier_list();
		match(TK_TPOINTS);
		type();
	} else {
		error();
	}
}
\end{minted}
\begin{minted}[autogobble,fontsize=\small,xleftmargin=1cm,tabsize=1]{java}
void statement_block() {
	switch (preanalisis) {
		case TK_ID:
		case TK_WRITE:
		case TK_READ:
		case TK_IF:
		case TK_WHILE:
			statement();
			break;
		case TK_BEGIN:
			multiple_statement();
			break;
	}
}
\end{minted}
\begin{minted}[autogobble,fontsize=\small,xleftmargin=1cm,tabsize=1]{java}
void multiple_statement() {
	if (preanalisis == TK_BEGIN)) {
		match(TK_BEGIN);
		statement_list();
		match(TK_END);
	} else {
		error();
	}
}
\end{minted}
\begin{minted}[autogobble,fontsize=\small,xleftmargin=1cm,tabsize=1]{java}
void statement_list() {
	switch (preanalisis) {
		case TK_ID:
		case TK_WRITE:
		case TK_READ:
		case TK_IF:
		case TK_WHILE:
			statement();
			statement_list_1();
			break;
		default:
			error();
			break;
	}
}
\end{minted}
\begin{minted}[autogobble,fontsize=\small,xleftmargin=1cm,tabsize=1]{java}
void statement_list_1() {
	if (preanalisis == TK_ENDSTNC)) {
		match(TK_ENDSTNC);
		statement_list();
	}
}
\end{minted}
\begin{minted}[autogobble,fontsize=\small,xleftmargin=1cm,tabsize=1]{java}
void statement() {
	switch (preanalisis) {
		case TK_ID:
		case TK_WRITE:
		case TK_READ:
			simple_statement();
			break;
		case TK_IF:
		case TK_WHILE:
			structured_statement();
			break;
		default:
			error();
			break;
	}
}
\end{minted}
\begin{minted}[autogobble,fontsize=\small,xleftmargin=1cm,tabsize=1]{java}
void simple_statement() {
	switch (preanalisis) {
		case TK_ID:
			identifier();
			simple_statement_1();
			break;
		case TK_WRITE:
			match(TK_WRITE);
			call_procedure_or_function();
			break;
		case TK_READ:
			match(TK_READ);
			call_procedure_or_function();
			break;
		default:
			error();
			break;
	}
}
\end{minted}
\begin{minted}[autogobble,fontsize=\small,xleftmargin=1cm,tabsize=1]{java}
void simple_statement_1() {
	switch (preanalisis) {
		case TK_ASSIGN:
			assignment_statement();
			break;
		case TK_OPAR:
			call_procedure_or_function();
			break;
	}
}
\end{minted}
\begin{minted}[autogobble,fontsize=\small,xleftmargin=1cm,tabsize=1]{java}
void structured_statement() {
	switch (preanalisis) {
		case TK_IF:
			conditional_statement();
			break;
		case TK_WHILE:
			repetitive_statement();
			break;
		default:
			error();
			break;
	}
}
\end{minted}
\begin{minted}[autogobble,fontsize=\small,xleftmargin=1cm,tabsize=1]{java}
void assignment_statement() {
	if (preanalisis == TK_ASSIGN)) {
		match(TK_ASSIGN);
		expression_or();
	} else {
		error();
	}
}
\end{minted}
\begin{minted}[autogobble,fontsize=\small,xleftmargin=1cm,tabsize=1]{java}
void call_procedure_or_function() {
	if (preanalisis == TK_OPAR)) {
		match(TK_OPAR);
		call_procedure_or_function_1();
		match(TK_CPAR);
	} else {
		error();
	}
}
\end{minted}
\begin{minted}[autogobble,fontsize=\small,xleftmargin=1cm,tabsize=1]{java}
void call_procedure_or_function_1() {
	switch (preanalisis) {
		case TK_ID:
		case TK_OPAR:
		case TK_ADD_OP_REST:
		case TK_NOT_OP:
		case TK_BOOLEAN_TRUE:
		case TK_BOOLEAN_FALSE:
		case TK_NUMBER:
			expression_list();
			break;
	}
}
\end{minted}
\begin{minted}[autogobble,fontsize=\small,xleftmargin=1cm,tabsize=1]{java}
void conditional_statement() {
	if (preanalisis == TK_IF)) {
		match(TK_IF);
		expression_or();
		match(TK_THEN);
		statement_block();
		else_statement();
	} else {
		error();
	}
}
\end{minted}
\begin{minted}[autogobble,fontsize=\small,xleftmargin=1cm,tabsize=1]{java}
void else_statement() {
	if (preanalisis == TK_ELSE)) {
		match(TK_ELSE);
		statement_block();
	}
}
\end{minted}
\begin{minted}[autogobble,fontsize=\small,xleftmargin=1cm,tabsize=1]{java}
void repetitive_statement() {
	if (preanalisis == TK_WHILE)) {
		match(TK_WHILE);
		expression_or();
		match(TK_DO);
		statement_block();
	} else {
		error();
	}
}
\end{minted}
\begin{minted}[autogobble,fontsize=\small,xleftmargin=1cm,tabsize=1]{java}
void expression_list() {
	switch (preanalisis) {
		case TK_ID:
		case TK_OPAR:
		case TK_ADD_OP_REST:
		case TK_NOT_OP:
		case TK_BOOLEAN_TRUE:
		case TK_BOOLEAN_FALSE:
		case TK_NUMBER:
			expression_or();
			expression_list_1();
			break;
		default:
			error();
			break;
	}
}
\end{minted}
\begin{minted}[autogobble,fontsize=\small,xleftmargin=1cm,tabsize=1]{java}
void expression_list_1() {
	if (preanalisis == TK_COMMA)) {
		match(TK_COMMA);
		expression_list();
	}
}
\end{minted}
\begin{minted}[autogobble,fontsize=\small,xleftmargin=1cm,tabsize=1]{java}
void expression_or() {
	switch (preanalisis) {
		case TK_ID:
		case TK_OPAR:
		case TK_ADD_OP_REST:
		case TK_NOT_OP:
		case TK_BOOLEAN_TRUE:
		case TK_BOOLEAN_FALSE:
		case TK_NUMBER:
			expression_and();
			expression_or_1();
			break;
		default:
			error();
			break;
	}
}
\end{minted}
\begin{minted}[autogobble,fontsize=\small,xleftmargin=1cm,tabsize=1]{java}
void expression_or_1() {
	if (preanalisis == TK_OR)) {
		match(TK_OR);
		expression_and();
		expression_or_1();
	}
}
\end{minted}
\begin{minted}[autogobble,fontsize=\small,xleftmargin=1cm,tabsize=1]{java}
void expression_and() {
	switch (preanalisis) {
		case TK_ID:
		case TK_OPAR:
		case TK_ADD_OP_REST:
		case TK_NOT_OP:
		case TK_BOOLEAN_TRUE:
		case TK_BOOLEAN_FALSE:
		case TK_NUMBER:
			expression_rel();
			expression_and_1();
			break;
		default:
			error();
			break;
	}
}
\end{minted}
\begin{minted}[autogobble,fontsize=\small,xleftmargin=1cm,tabsize=1]{java}
void expression_and_1() {
	if (preanalisis == TK_AND)) {
		match(TK_AND);
		expression_rel();
		expression_and_1();
	}
}
\end{minted}
\begin{minted}[autogobble,fontsize=\small,xleftmargin=1cm,tabsize=1]{java}
void expression_rel() {
	switch (preanalisis) {
		case TK_ID:
		case TK_OPAR:
		case TK_ADD_OP_REST:
		case TK_NOT_OP:
		case TK_BOOLEAN_TRUE:
		case TK_BOOLEAN_FALSE:
		case TK_NUMBER:
			expression_add();
			expression_rel_1();
			break;
		default:
			error();
			break;
	}
}
\end{minted}
\begin{minted}[autogobble,fontsize=\small,xleftmargin=1cm,tabsize=1]{java}
void expression_rel_1() {
	switch (preanalisis) {
		case TK_REL_OP_EQ:
		case TK_REL_OP_NEQ:
		case TK_REL_OP_MIN:
		case TK_REL_OP_MAY:
		case TK_REL_OP_LEQ:
		case TK_REL_OP_GEQ:
			relational_operator();
			expression_add();
			expression_rel_1();
			break;
	}
}
\end{minted}
\begin{minted}[autogobble,fontsize=\small,xleftmargin=1cm,tabsize=1]{java}
void expression_add() {
	switch (preanalisis) {
		case TK_ID:
		case TK_OPAR:
		case TK_ADD_OP_REST:
		case TK_NOT_OP:
		case TK_BOOLEAN_TRUE:
		case TK_BOOLEAN_FALSE:
		case TK_NUMBER:
			expression_mult();
			expression_add_1();
			break;
		default:
			error();
			break;
	}
}
\end{minted}
\begin{minted}[autogobble,fontsize=\small,xleftmargin=1cm,tabsize=1]{java}
void expression_add_1() {
	switch (preanalisis) {
		case TK_ADD_OP_SUM:
		case TK_ADD_OP_REST:
			addition_operator();
			expression_mult();
			expression_add_1();
			break;
	}
}
\end{minted}
\begin{minted}[autogobble,fontsize=\small,xleftmargin=1cm,tabsize=1]{java}
void expression_mult() {
	switch (preanalisis) {
		case TK_ID:
		case TK_OPAR:
		case TK_ADD_OP_REST:
		case TK_NOT_OP:
		case TK_BOOLEAN_TRUE:
		case TK_BOOLEAN_FALSE:
		case TK_NUMBER:
			factor();
			expression_mult_1();
			break;
		default:
			error();
			break;
	}
}
\end{minted}
\begin{minted}[autogobble,fontsize=\small,xleftmargin=1cm,tabsize=1]{java}
void expression_mult_1() {
	switch (preanalisis) {
		case TK_MULT_OP_POR:
		case TK_MULT_OP_DIV:
			multiplication_operator();
			factor();
			expression_mult_1();
			break;
	}
}
\end{minted}
\begin{minted}[autogobble,fontsize=\small,xleftmargin=1cm,tabsize=1]{java}
void factor() {
	switch (preanalisis) {
		case TK_ID:
			identifier();
			factor_1();
			break;
		case TK_OPAR:
			match(TK_OPAR);
			expression_or();
			match(TK_CPAR);
			break;
		case TK_ADD_OP_REST:
		case TK_NOT_OP:
			unary_operator();
			factor();
			break;
		case TK_BOOLEAN_TRUE:
		case TK_BOOLEAN_FALSE:
		case TK_NUMBER:
			literal();
			break;
		default:
			error();
			break;
	}
}
\end{minted}
\begin{minted}[autogobble,fontsize=\small,xleftmargin=1cm,tabsize=1]{java}
void factor_1() {
	if (preanalisis == TK_OPAR)) {
		call_procedure_or_function();
	}
}
\end{minted}
\begin{minted}[autogobble,fontsize=\small,xleftmargin=1cm,tabsize=1]{java}
void relational_operator() {
	switch (preanalisis) {
		case TK_REL_OP_EQ:
			match(TK_REL_OP_EQ);
			break;
		case TK_REL_OP_NEQ:
			match(TK_REL_OP_NEQ);
			break;
		case TK_REL_OP_MIN:
			match(TK_REL_OP_MIN);
			break;
		case TK_REL_OP_MAY:
			match(TK_REL_OP_MAY);
			break;
		case TK_REL_OP_LEQ:
			match(TK_REL_OP_LEQ);
			break;
		case TK_REL_OP_GEQ:
			match(TK_REL_OP_GEQ);
			break;
		default:
			error();
			break;
	}
}
\end{minted}
\begin{minted}[autogobble,fontsize=\small,xleftmargin=1cm,tabsize=1]{java}
void unary_operator() {
	switch (preanalisis) {
		case TK_ADD_OP_REST:
			match(TK_ADD_OP_REST);
			break;
		case TK_NOT_OP:
			match(TK_NOT_OP);
			break;
		default:
			error();
			break;
	}
}
\end{minted}
\begin{minted}[autogobble,fontsize=\small,xleftmargin=1cm,tabsize=1]{java}
void addition_operator() {
	switch (preanalisis) {
		case TK_ADD_OP_SUM:
			match(TK_ADD_OP_SUM);
			break;
		case TK_ADD_OP_REST:
			match(TK_ADD_OP_REST);
			break;
		default:
			error();
			break;
	}
}
\end{minted}
\begin{minted}[autogobble,fontsize=\small,xleftmargin=1cm,tabsize=1]{java}
void multiplication_operator() {
	switch (preanalisis) {
		case TK_MULT_OP_POR:
			match(TK_MULT_OP_POR);
			break;
		case TK_MULT_OP_DIV:
			match(TK_MULT_OP_DIV);
			break;
		default:
			error();
			break;
	}
}
\end{minted}
\begin{minted}[autogobble,fontsize=\small,xleftmargin=1cm,tabsize=1]{java}
void type() {
	switch (preanalisis) {
		case TK_TYPE_INT:
			match(TK_TYPE_INT);
			break;
		case TK_TYPE_BOOL:
			match(TK_TYPE_BOOL);
			break;
		default:
			error(un tipo de dato);
			break;
	}
}
\end{minted}
\begin{minted}[autogobble,fontsize=\small,xleftmargin=1cm,tabsize=1]{java}
void identifier_list() {
	if (preanalisis == TK_ID)) {
		identifier();
		identifier_list_1();
	} else {
		error();
	}
}
\end{minted}
\begin{minted}[autogobble,fontsize=\small,xleftmargin=1cm,tabsize=1]{java}
void identifier_list_1() {
	if (preanalisis == TK_COMMA)) {
		match(TK_COMMA);
		identifier_list();
	}
}
\end{minted}
\begin{minted}[autogobble,fontsize=\small,xleftmargin=1cm,tabsize=1]{java}
void identifier() {
	if (preanalisis == TK_ID)) {
		match(TK_ID);
	} else {
		error();
	}
}
\end{minted}
\begin{minted}[autogobble,fontsize=\small,xleftmargin=1cm,tabsize=1]{java}
void literal() {
	switch (preanalisis) {
		case TK_BOOLEAN_TRUE:
		case TK_BOOLEAN_FALSE:
			bool();
			break;
		case TK_NUMBER:
			number();
			break;
		default:
			error();
			break;
	}
}
\end{minted}
\begin{minted}[autogobble,fontsize=\small,xleftmargin=1cm,tabsize=1]{java}
void number() {
	if (preanalisis == TK_NUMBER)) {
		match(TK_NUMBER);
	} else {
		error();
	}
}
\end{minted}
\begin{minted}[autogobble,fontsize=\small,xleftmargin=1cm,tabsize=1]{java}
void bool() {
	switch (preanalisis) {
		case TK_BOOLEAN_TRUE:
			match(TK_BOOLEAN_TRUE);
			break;
		case TK_BOOLEAN_FALSE:
			match(TK_BOOLEAN_FALSE);
			break;
		default:
			error();
			break;
	}
}
\end{minted}
\end{multicols}

\end{document}